%-------------------------------------------------------------------------------
%	SECTION TITLE
%-------------------------------------------------------------------------------
\cvsection{Experience}


%-------------------------------------------------------------------------------
%	CONTENT
%-------------------------------------------------------------------------------
\begin{cventries}

%---------------------------------------------------------
  \cventry
    {R\&D Engineer} % Job title
    {Kitware Inc.} % Organization
    {Clifton Park, NY} % Location
    {Jun. 2016 - Now} % Date(s)
    {
      \begin{cvitems} % Description(s) of tasks/responsibilities
        \item {\fontsize{10pt}{1em}\bodyfont\bfseries\color{darktext}Scientific multi-threading algorithms for Massively Threaded Architectures toolkit(VTK-m)}
          \begin{itemize}
              \item {Implemented a bunch of parallel scientific algorithms(probe, histogram, etc) and improved the shared memory support for CUDA and runtime selection for TBB and OpenMP}
              \item {Simplified and optimized the testing infrastructure which saved runtime overheads by CRTP pattern and generic programming. It reduced the binary size by 80\% and improved 40\% of runtime performance}
              \item {Refactored the code base with C++11 features which simplified meta programming, improving runtime overhead as well as adding a reliable random number generation and prepare for C++14 migration}
               % thread_local
              \item {Added an address sanitizer nightly build and automated the uploading process to a remote nightly dashboard which catches more than 30 legacy bugs}
          \end{itemize}
        \item {\fontsize{10pt}{1em}\bodyfont\bfseries\color{darktext}Adaptable Input Output System Version 2(ADIOS2)}
          \begin{itemize}
              % zero-copy model
            \item {Implemented a parallel data reader which could distribute and balance loads among many processes and visualize streaming data, in-situ data in large scale. It's capable of generating unstructured grid, uniform grid, polydata and volume data}
              \item {Parallelized the engine tests(BP, HDF5, ADIOS1, etc) and binding tests(Python, Fortran), tested on hundreds of processor cores}
              \item {Extended FindGoogleTesting CMake module so that it can launch MPI tests and provide a better CMake integration}
          \end{itemize}
        \item {\fontsize{10pt}{1em}\bodyfont\bfseries\color{darktext}Simulation Modeling ToolKit and Computational Model Builder framework}
          \begin{itemize}
              \item {Architected a simulation tool kit which provideded a front end UI to design and visualize complicated model via Qt and GPU glyphing, adding C++11 python binding support and ability to launch parallel meshing jobs via a thread pool with commerical meshing libraries}
              \item {Created a meteorological geo-data browser which allow users to fetch data asynchronizally from remote servers and analyze them on real world map via OpenStreetMap RESTful API and MVC design pattern}
              \item {Profiled the existing framework and redesigned its data I\/O serialization via nlohmann\_json which improved the performance by 20\%}
              \item {Created a singleton model manager which bridged the modeling server and visualization client}
          \end{itemize}
      \end{cvitems}
    }

%---------------------------------------------------------
%  \cventry
%    {Research Assistant} % Job title
%    {Visual Engineer and Enginnering Lab@CMU} % Organization
%    {Pittsburgh, PA} % Location
%    {Dec. 2014 - May. 2016} % Date(s)
%    {
%      \begin{cvitems} % Description(s) of tasks/responsibilities
%        \item {Human Tumor Deformation Prediction - A NIH funded project}
%        \begin{itemize}
%          \item {Based on fiducial marker info, make a prediction on tumor global deformation via Ansys APDL, Matlab to help surgeons improve operation accuracy}
%          \item {Built a software based on OpenFlipper, TetGen and optimization packages to enable I/O, tetrahedron volume mesh generation and visualization for medical usages}
%        \end{itemize}
%        \item {Design for 3D Printing through Spatial partitioning and Assembly for large models}
%        \begin{itemize}
%          \item {Constructed a beam search based framework to automatically partition the surface mesh model and output the segmented parts}
%          \item {Utilized PCA to segment the model and optimized overall 3D printing time up to 40\%}
%        \end{itemize}
%      \end{cvitems}
%    }

%---------------------------------------------------------
\end{cventries}
